\documentclass{beamer}

\usetheme{ClassyCharcoal}
\usefonttheme{structurebold}
\newenvironment{witemize}{\itemize\addtolength{\itemsep}{15pt}}{\enditemize}

\usepackage{graphicx}

\begin{document}


%%--------- Title ------------

\title{Lecture 1: Overview of Machine Learning}
\author{Dahua Lin}
\institute{The Chinese University of Hong Kong}
\date{}
\frame{\titlepage}


%%---------  Main Body ------------

\begin{frame}{About this course}

\begin{witemize}
    \item This is a \emph{graduate level} introduction to \emph{statistical learning}. 
    \item The course is \emph{not} to teach you:
    \begin{itemize}
        \item Support Vector Machine
        \item Linear Regression
        \item ...
        \item Deep Learning
    \end{itemize}
    \item Instead, you are going to learn \emph{foundational tools} for developing \emph{your own} models and algorithms.
\end{witemize}
\end{frame}


% ---


\begin{frame}{Course Format}

\begin{witemize}
    \item \structure{No Exams!}
    \item Topic driven
    \item For each topic:
    \begin{itemize}
        \item Introductory lecture
        \item Paper reading and Homework
        \item In-class discussion
    \end{itemize}
    \item You will present a paper/subject at the end of the course
\end{witemize}
\end{frame}


% ---


\begin{frame}{}

{\Large \em What is \structure{Machine Learning}?}

$ $
\pause

\begin{quote}
    \emph{Machine learning} is a scientific discipline that explores the construction and study 
    of algorithms that can learn from data. Such algorithms operate by building a model based on 
    inputs and using that to make predictions and decisions, rather than following only explicitly 
    programmed instructions.   -- Wikipedia
\end{quote}
\end{frame}


% ---


\begin{frame}{Elements of Machine Learning}
\begin{witemize}
    \item Elements:
    \begin{itemize}
        \item Data
	    \item Model
        \item Learning Algorithm
        \item Prediction
    \end{itemize}
    \item Learn from old data, make predictions on new data. 
    The common aspects of both the old and new data are captured by the model.
\end{witemize}
\end{frame}


% ---

\begin{frame}{}

Please write down five machine learning algorithms that you know

$ $

Don't write \emph{Deep Learning}
\end{frame}


% ---


\begin{frame}{Overview of Machine Learning}
\begin{witemize}
    \item Tasks
    \begin{itemize}
        \item Supervised learning
	    \item Unsupervised learning
	    \item Semi-supervised learning
	    \item Reinforcement learning
    \end{itemize}
    \item Problems
    \begin{itemize}
        \item Classification
	    \item Regression
	    \item Clustering
	    \item Dimension reduction
	    \item Density estimation
    \end{itemize}
\end{witemize}
\end{frame}


% ---


\begin{frame}{Topics}
\begin{itemize}
    \item Exponential family distributions and conjugate prior
    \item Generalized linear model
    \item Empirical risk minimization and Stochastic gradient descent 
    \item Proximal methods for optimization
    \item Graphical models: Bayesian Networks and Markov random fields
    \item Sum-product and max-product algorithms, Belief propagation
    \item Variational inference methods
    \item Markov Chain Monte Carlo
    \item Gaussian Processes and Copula Processes
    \item Handling Big Data: Streaming process and Core sets
\end{itemize}
\end{frame}


% ---


\begin{frame}{}
\begin{center}
Let's take a five-minute break.
\end{center}
\end{frame}









\end{document}
